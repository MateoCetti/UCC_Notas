\title{Resumen de filosofia}

\author{Mateo P. Cetti}

\documentclass[10pt]{article}

\usepackage{amsmath}
\usepackage{amsfonts}

\begin{document} 
\maketitle

\paragraph{comienzo de la filosofia}
lA FILOSOFIA COMO UN DISCURSO SISTEMATIZADO COMENZO EN GRECIA (version eurocentrica).\\
Todo pueblo y toda cultura tiene un saber a partir del cual organiza su vida y relaciones sociales.\\
\linebreak
El mito explica verdades en otro lenguaje\\
\linebreak
La filosofia no nace en grecia porque eran inteligentes.\\
\linebreak
El contexto historico influye en nuestra manera de pensar.\\
\linebreak
Cuando se transforma la matriz economica, generalmente se transforma tambien la matriz cultural y social.\\
\linebreak
Uno de los cambios socio-culturales mas importantes son el paso de la oralidada a la escritura (para mitos en grecia). Esto repercutio de manera positiva para la filosofia, porque la memoria se libera y las personas incorporan ideas novedosas (ideas de otras culturas, conocimientos cientificos, ideas de el alma, etc).\\
\linebreak
El paso de las mitologias a las cosmologias "Cientificas".

\paragraph{Distincion entre filosofia, pensamiento cotidiano y pensamiento cientifico}
\paragraph{Tres definiciones de filosofia}
\textbf{Experiencia reflexionada} Toda experiencia humana que pasa por la reflexion puede ser conciderada un pensamiento filosofico (sufrimiento, fracaso, desamor, traicion, etc).\\
\linebreak
\textbf{Continuacion de lucha politica}
%


\end{document}