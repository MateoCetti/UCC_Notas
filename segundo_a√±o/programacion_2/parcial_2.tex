\documentclass[11pt]{article}

\usepackage{sectsty}
\usepackage{amsmath}
\usepackage{graphicx}

% Margins
\topmargin=-0.45in
\evensidemargin=0in
\oddsidemargin=0in
\textwidth=6.5in
\textheight=9.0in
\headsep=0.25in

\title{ Resumen programación 2$^{do}$ parcial}
\author{ Mateo P. Cetti }
\date{\today}

\begin{document}
\maketitle	
\pagebreak

% Optional TOC
% \tableofcontents
% \pagebreak

%--Paper--

\section{Paradigmas y lenguajes de programacion}
\paragraph{Lenguaje de programacion}
Cualquier lenguaje artificial que pueda utilizarse para definir una secuencia de instrucciones para su procesamiento
en un ordenador\\
\textbf{Caracteristicas}
\begin{enumerate}
	\item Define un proceso que se ejecuta en un ordenador
	\item Es de alto nivel, cercano a los problemas q se quieren resolver
	\item Permite construir nuevas abstracciones que se adapten al dominio que se programa
\end{enumerate}
\textbf{Elementos}
\begin{enumerate}
	\item Expreciones primitivas que representan las entidades mas simples del lenguaje
	\item Mecanismos de combinacion con los que se construyen elementos compuestos a partir de elementos mas simples
	\item Mecanismos de abstraccion con los que dar nombre a los elementos compuestos y manipularlos como unidades.
\end{enumerate}
\paragraph{Paradigmas de programacion}
Define un conjunto de reglas, patrones y estilos de programacion que son usados por un grupo de lenguajes de programacion
\begin{enumerate}
	\item \textbf{Paradigma funcional} El concepto de estado o \textbf{variable} no existe. se basa en el concepto de funcion
		que describe una relacion entre una entrada y una salida.\\
		Ejemplos: Haskell, Scala.
		\item \textbf{Paradigma logico} Los problemas se modelan usando la logica formal llegando a una conclucion por medio de 			hechos y reglas\\
		Ejemplos: Prolog, mercury.
		\item \textbf{Paradigma imperativo} La computacion se realiza cambiando el estado del programa por medio de sentencias 			que definen pasos de ejecucion.\\
		Ejemplos: C, Basic, js.
		\item \textbf{Paradigma orientado a objetos} Se basa en la idea natural de un mundo lleno de objetos
				y la resolucion de problemas se realiza mediante el modelado de esos objetos.
		\textbf{Caracteristicas:}		
		\begin{itemize}
			\item Definicion de clases y herencia
			\item Objetos como abstraccion de datos y procedimientos
			\item Polimorfismo y chequeo de tipos en tiempo de ejecucion.
		\end{itemize}
		Ejemplos: C++, Java, PhP.
\end{enumerate}
\paragraph{Interpretes} Dado un programa fuente, en un lenguaje de alto nivel, realiza la traduccion y ejecucion (de forma simultanea) instruccion a instruccion.\\
Ejemplos: Python, Ruby, PHP.
\paragraph{Compiladores} Dado un programa fuente en un lenguaje de alto nivel, se realiza la traduccion \textbf{completa} generando un programa objeto, convertido en otro lenguaje.\\
Ejemplos: C, C++\\
\textbf{Etapas de la compilacion}
\begin{enumerate}
	\item \textbf{Edicion} Se escribe el programa. (Se obtiene el codigo fuente)
	\item \textbf{Compilacion} Se traduce el codigo fuente a codigo maquina
	\begin{itemize}
		\item Si no se produce ningun error se obtiene el codigo objeto
		\item Si se encuentra un error la traduccion no se realiza y se informa el error
	\end{itemize}
	\item \textbf{Linkeo} Se une el codigo objeto a las librerias del lenguaje (se obtiene el programa ejecutable).\\
	Existen 2 tipos de linkeo:
	\begin{itemize}
		\item \textbf{Linkeo dinamico} El codigo objeto de las librerias se añaden al codigo objeto del programa generando el archivo ejecutable.
		\item \textbf{Linkeo estatico} El código objeto de las librerías no se añade al código objeto del programa, se cargan y se ligan en el momento en que se necesiten.
	\end{itemize}
\end{enumerate}
\section{Punteros} Variables que almacenan \textbf{direcciones de memoria} de un tipo de variable determinado (int, double, etc).
\paragraph{Declaracion de punteros} int *p\\
\textbf{acceder al valor de un puntero} \&p \\
\textbf{Memoria dinamica} \\
int *v;\\
v = new int[100];\\
\section{Programacion orientada a objetos}
\paragraph{}Facilita la creacion de software de calidad.\\
Sus caracteristicas potencian la:\\
\begin{itemize}
	\item Mantencion
	\item Reutilizacion
	\item extencion
\end{itemize}
del software.
\paragraph{Abstraccion} El modelo define una \textbf{perspectiva abstracta} del problema, con \textbf{Datos} y \textbf{operaciones}
\paragraph{Conceptos de POO}
\begin{enumerate}
	\item \textbf{Clase} Posee 2 categorias de datos: \textbf{Atributos} y \textbf{Metodos}
	\item \textbf{Instancias/objetos} es una ocurrencia de una clase especifica. Al momento de crear un objeto se produce la \textbf{instanciacion}
	\item \textbf{Atributos} Datos que caracterizan al objeto y definen su estado
	\item \textbf{Metodos} Representan acciones que se pueden realizar sobre un objeto. son \textbf{funciones}.
\end{enumerate}
\paragraph{Principios del POO}
Principalmente $\rightarrow$ codigo reutilizable 
\paragraph{Encapsulamiento:}proceso por el cual se \textbf{ocultan} las estructuras de datos y los detalles de la implementacion. los atributos y metodos estan encapsulados (contenidos) dentro de la clase.
\paragraph{Herencia}
\paragraph{Polimorfismo}
\paragraph{Objetos}
Los objetos constan de:
\begin{itemize}
	\item \textbf{Tiempo de vida} duracion definida de un objeto en un programa
	\item \textbf{Estado} Queda definido por sus atributos
	\item \textbf{Comportamiento} Queda definido por sus metodos
\end{itemize}
\paragraph{Clase} son \textbf{abstracciones} que representan a un conjunto de objetos
\paragraph{Mensaje} Mecanismo por el cual se solicita una accion sobre el objeto\\
la interaccion entre objetos se basa en \textbf{mensajes} de modo que el emisor le pide al receptor la ejecucion de un metodo.
\paragraph{Interfaz} Mecanismo mediante el cual un objeto puede comunicarse con el medio. Se materializa mediante los metodos publicos de una clase
\paragraph{constructores} Tienen el mismo nombre de la clase, no returnean nada, pueden ser sobrecargados.
\paragraph{Destructores}
\begin{itemize}
	\item libera los recursos solicitados por el constructor
	\item No se requiere utilizar si no se utiliza almacenamiento dinamico
	\item Se invocan implicitamente cuando finaliza el bloque en el que fue declarado el objeto
\end{itemize}
\section{Sobrecarga de operadores}
Es posible \textbf{redefinir} algunos \textbf{OPERADORES} para los objetos de una clase para simplificar el codigo al maximo.\\
se pueden sobrecargar casi todos los operadores unitarios (cardinalidad 1) o binarios (cardinalidad 2) y el operador llamado a funcion \textbf{()}\\
Se puede modificar la \textbf{definicion} de un operador pero no su \textbf{gramatica} (numero de operandos, precedencia y asociatividad) ni la \textbf{cardinalidad}.
\section{Herencia}
\begin{itemize}
	\item Definir una nueva clase \textbf{hija} a partir de una clase \textbf{padre}.
	\item los ejemplares de la subclase pueden acceder a los miembros (Atributos, metodos) de la superclase.
	\item Herencia es transitiva
	\item Una subclase tiene todas las propiedades de la superclase y otras más (extensión)
	\item Una subclase constituye una especialización de la superclase (reducción)
	\item Un método de superclase es anulado por un método con el mismo nombre definido en la subclase
	\item Un constructor de subclase siempre invoca primero al constructor de su superclase 
	\item Un destructor de subclase se ejecuta antes que el destructor de su superclase
	\item No se necesita reescribir el código del comportamiento heredado
	\item las relaciones de herencia forman una jerarquia entre clases
	\item \textbf{Herencia múltiple} class CuentaEmpresarial: public Cuenta, public Empresa
	\item Hay algunos elementos de la clase base que no pueden ser heredados:
	\begin{itemize}
		\item Constructores
		\item Destructores
		\item Funciones y datos estáticos de la clase
		\item Operador de asignación (=) sobrecargado
	\end{itemize}
\end{itemize}
\section{Polimorfismo}
\paragraph{Funciones virtuales} son funciones distintas con el mismo nombre. declaradas \textit{virtual} en la clase base
\paragraph{Funciones virtuales puras} La funcion de la clase base debe declararse, aunque no hace falta definirla. no se pueden instanciar objetos de esa clase pero si punteros. si una clase tiene una o mas funciones virtuales puras, esta se convierte en una \textbf{clase abstracta}\\
Si una clase derivada no define una funcion virtual pura, la hereda como pura y por lo tanto tambien sera abstracta.\\
Una clase que define todas las funciones virtuales es una clase \textbf{concreta}
\section{Composicion de clases}
\begin{itemize}
	\item Consiste en declarar objetos de una clase A como \textbf{atributos} de una clase B. 
	\item El constructor de la case B llamara a los constructores de la clase A.
	\item Un constructor \textbf{por defecto} de una clase llamara implicitamente a los constructores por defecto de los 				objetos declarados como atributos.
\end{itemize}
\section{Templates}
Una plantilla (\textbf{Template}) es un patron para crear funciones y clases. Esto sirve para evitar escribir multiples versiones de la misma funcion para llevar a cabo la misma operacion sobre datos de \textbf{DISTINTO TIPO}\\
\paragraph{funciones template}
Las funciones template pueden tener 1 o mas parametros. para declarar la funcion template se utiliza el parametro T, que indica el \textbf{tipo de dato} que sera suministrado cuando se llame la funcion.\\
\paragraph{clases template}
Las plantillas de clases permiten crear nuevas clases con tipos de datos no definidos.
\section{Manejo de excepciones}
\paragraph{Excepcion} Errores que se producen durante la ejecucion.\\
Si se producen y no se manejan adecuadamente, el programa terminara abruptamente.\\
Si se implementa manejo de excepciones, la calidad de los programas aumentara conciderablemente.\\
Este manejo consiste en transferir la ejecucion del programa desde un punto donde se produce la excepcion a un manipulador que \textbf{coincida} con el motivo de la excepcion.\\
\textbf{Pasos:}
\begin{enumerate}
	\item se \textbf{intenta} ejecutar un bloque de codigo y se decide que hacer si se produce un error.
	\item se produce o se \textbf{lanza} ese error
	\item La ejecucion del programa es \textbf{desviada} a un sitio especifico donde la excepcion es \textbf{capturada} y se decide que hacer al respecto.
\end{enumerate}
\end{document}