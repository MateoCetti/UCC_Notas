\title{Resumen de Fisica 3}

\author{Mateo P. Cetti}

\documentclass[10pt]{article}

\usepackage{amsmath}
\usepackage{amsfonts}

\begin{document} 
\maketitle

\section{Movimiento ondulatorio}

El mundo está lleno de \textbf{ondas}, los dos tipos principales son las ondas \textbf{mecánicas} 
(Tirar una piedra al agua) y las ondas \textbf{electromagnéticas} (las ondas de radio). En el caso de las 
ondas mecánicas, algunos medios físicos se perturban. Las ondas
electromagnéticas no requieren un medio para propagarse.\\
La caracteristica principal del movimiento ondulatorio es que la \textbf{energía} se 
transfiere a través de una distancia, perola \textbf{materia} no.

\paragraph{Ondas mecanicas}

Todas las ondas mecánicas requieren:

\begin{enumerate}
    \item alguna fuente de perturbación
    \item un medio que contenga elementos que sean factibles de perturbación
    \item algún mecanismo físico a partir del cual los elementos del medio puedan influirse mutuamente.
\end{enumerate}

\paragraph{Tipos de ondas: }

\begin{itemize}
    \item \textbf{onda transversal} Una onda viajera o pulso que hace que los elementos 
    del medio perturbado se muevan \textbf{perpendiculares} a la dirección de propagación.
    \item \textbf{onda longitudinal} Una onda viajera o pulso que mueve a los elementos del medio en \textbf{paralelo}
    a la dirección de propagación
\end{itemize}

\paragraph{Funcion de onda}
\begin{itemize}
    \item $y(x, t) y(x-vt, 0)$ (Si el pulso viaja hacia la \textbf{derecha} en X)
    \item $y(x, t) y(x+vt, 0)$ (Si el pulso viaja hacia la \textbf{izquierda} en X)
\end{itemize}

La función de onda y(x, t) representa la coordenada y, la posición transversal,
de cualquier \textbf{elemento} ubicado en la posición x en cualquier tiempo t.\\
\linebreak
Ademas, si t es fijo (como en el caso de tomar una instantánea del pulso), la función 
de onda y(x), a veces llamada \textbf{forma de onda}, define una curva que representa la 
\textbf{forma} geométrica del \textbf{pulso} en dicho tiempo.

\paragraph{El modelo de onda progresiva}

Primero, la forma de onda completa se mueve hacia la derecha de modo que la curva 
se mueve hacia la derecha y al final llega a la posición de la curva azul. Este es el \textbf{movimiento 
de la onda}. Si se concentra en un elemento del medio, como el elemento en x = 0, observará que cada
elemento se mueve hacia arriba y hacia abajo a lo largo del eje y en movimiento armónico
simple. Este es el \textbf{movimiento de los elementos del medio}. Es importante \textbf{diferenciar} entre el
movimiento de la onda y el movimiento de los elementos del medio.

\paragraph{modelo de onda}

Un punto en la figura en que el desplazamiento del elemento de su 
posición normal está más alto se llama \textbf{cresta} de la onda.\\

El punto más bajo se llama \textbf{valle}.\\

la longitud de onda ($\lambda$) es la \textbf{distancia mínima} entre dos puntos
cualesquiera en ondas adyacentes\\

\textbf{periodo} T de las ondas es el intervalo de tiempo requerido 
para que dos puntos idénticos de ondas adyacentes pasen por un punto.\\

\textbf{frecuencia} $f$ de una onda periódica es el número de crestas (o
valles o cualquier otro punto en la onda) que pasa un punto determinado en un intervalo
de tiempo \textbf{unitario}.\\

\begin{equation*}
    f = \dfrac{1}{T}
\end{equation*}

La \textbf{máxima posición} de un elemento del medio relativo a su \textbf{posición de equilibrio} se
llama \textbf{amplitud} A de la onda.\\

Las ondas viajan con una \textbf{rapidez} específica, y esta rapidez depende de las \textbf{propiedades
del medio perturbado}.\\

\paragraph{Formas de la funcion de onda:}
Forma 1:
\begin{equation*}
    y(x, t) =  A sen \left[ \frac{2\pi}{\lambda} (x-vt) \right] 
\end{equation*}

$v = \dfrac{\lambda}{T}$. Sustituyendo esto en la ecuacion anterior:

\begin{equation*}
    y = Asen \left[ 2\pi \left( \frac{x}{\lambda} - \frac{t}{T} \right) \right] 
\end{equation*}

Para abreviar la formula definimos:\\
\linebreak

\textbf{número de onda angular: }$k = \dfrac{2\pi}{\lambda}$

\textbf{frecuencia angular: }$w = \dfrac{2\pi}{T} = 2\pi f$\\
\linebreak
Entonces la funcion queda:

\begin{equation*}
    y = Asen(kx-wt)
\end{equation*}

Ademas, $v$ se puede definir alternativamente como: $v = \frac{w}{f} = \lambda f$\\
\linebreak
La función de onda conocida en la ecuación supone que la posición vertical y de
un elemento del medio es cero en $x=0$ y $t=0$. Este no necesita ser el caso. Si no lo es, la
función de onda por lo general se expresa en la forma:\\
$y = Asen(kx-wt + \phi)$\\
donde $\phi$ es la \textbf{constante de fase}

\paragraph{Ondas sinusoiodales en cuerdas}

Se puede usar esta expresión para describir el movimiento de cualquier elemento de la
cuerda. Un elemento en el punto P se mueve sólo verticalmente, y de este modo su coordenada x 
permanece constante. Por lo tanto, la rapidez transversal $v_y$ (no confundir con la 
rapidez de onda v) y la aceleración transversal $a_y$ de los \textbf{elementos} de la cuerda son

\begin{equation*}
    v_y = -wA cos(kx-wt)
\end{equation*}

\begin{equation*}
    a_y = -w^2 A sen(kx-wt)
\end{equation*}

Los valores \textbf{máximos} de la rapidez transversal y la
aceleración transversal son simplemente los valores absolutos de los coeficientes de las
funciones coseno y seno:

\begin{equation*}
    v_y max = wA
\end{equation*}

\begin{equation*}
    a_y max = w^2 A
\end{equation*}

La \textbf{rapidez transversal} y la \textbf{aceleración transversal} de los elementos de la cuerda no llegan
\textbf{simultáneamente} a sus valores máximos. La rapidez transversal llega a su valor máximo
($wA$) cuando $y=0$, mientras que la magnitud de la aceleración transversal llega a su valor
máximo ($w^2A$) cuando $y= \pm A$

\paragraph{La rapidez de ondas en cuerdas}

En esta sección se determina la rapidez de un pulso transversal que viaja en una cuerda
tensa. Primero cabe mencionar que se espera que la \textbf{rapidez} de la onda \textbf{aumente} con una
\textbf{tensión creciente} y debe disminuir a medida que \textbf{aumente} la \textbf{masa por unidad de longitud} 
de la cuerda.\\
\linebreak
Si la tensión en la cuerda es $T$ y su masa por unidad
de longitud es $\mu$, la rapidez de onda es:

\begin{equation*}
    \sqrt{\dfrac{T}{\mu}}
\end{equation*}

\paragraph{Reflexión y transmisión}

El modelo de onda progresiva describe ondas que viajan a través de un medio uniforme sin
interactuar con algo más en el camino. Ahora se considerará cómo una onda progresiva
es afectada cuando \textbf{encuentra un cambio} en el medio\\
\linebreak
considere un pulso que viaja en una cuerda que está rígidamente unida a un soporte en un extremo, como en
la figura. Cuando el pulso alcanza el soporte, se presenta un cambio severo en el medio: la
cuerda termina. Como resultado, el pulso experimenta reflexión; es decir, el pulso se
mueve de regreso a lo largo de la cuerda en la dirección opuesta.Note que el pulso reflejado está invertido.\\
\linebreak
Ahora considere otro caso. Esta vez, el pulso llega al final de una cuerda que es libre
de moverse verticalmente, como en la figura 16.14. La tensión en el extremo libre se mantiene porque la cuerda está amarrada a un anillo de masa despreciable que tiene libertad
para deslizarse verticalmente sobre un poste uniforme sin fricción. De nuevo, el pulso
se refleja, pero esta vez no se invierte. \\
\linebreak
Para finalizar, considere una situación en la que la frontera es intermedia entre estos
dos extremos. En este caso, parte de la energía en el pulso incidente se refleja y parte se
somete a transmisión; es decir: parte de la energía pasa a través de la frontera. Por ejemplo, 
suponga que una cuerda ligera se une a una cuerda más pesada,

\paragraph{Rapidez de transferencia de energía mediante ondas sinusoidales
en cuerdas}

La energía cinética de un elemento de la cuerda se expresa como: 
\begin{equation*}
    dK = \frac{1}{2} (\mu dx) v_y^2
\end{equation*}

Al sustituir con la ecuación para la rapidez transversal general 
de un oscilador armónico simple se obtiene:

\begin{equation*}
    dK = \frac{1}{2} \mu w^2 A^2 cos^2 (kx-wt) dx 
\end{equation*}

Si se toma una instantánea de la onda en el tiempo t=0, la energía cinética de un elemento dado es:

\begin{equation*}
    dK = \frac{1}{2} \mu w^2 A^2 cos^2 (kx) dx 
\end{equation*}

Al integrar esta expresión sobre todos los elementos de cuerda en una longitud de onda
de la onda produce la energía cinética total KA en una longitud de onda:

\begin{equation*}
    dK_\lambda = \frac{1}{4} \mu w^2 A^2 \lambda 
\end{equation*}

para la energía potencial total $U_A$
en una longitud de onda produce exactamente el mismo resultado:

\begin{equation*}
    dU_\lambda = \frac{1}{4} \mu w^2 A^2 \lambda 
\end{equation*}

La energía total en una longitud de onda de la onda es la suma de las energías potencial
y cinética:

\begin{equation*}
    E_\lambda = U_\lambda + K_\lambda =  \frac{1}{4} \mu w^2 A^2 \lambda 
\end{equation*}

Por lo tanto, la potencia $\mathbb{P}$, o rapidez de transferencia de energía $T_{OM}$ asociada
con la onda mecánica, es

\begin{equation*}
    \mathbb{P} = \frac{1}{2} \mu w^2 A^2 v
\end{equation*}

\paragraph{La ecuación de onda lineal}

Todas las funciones de onda y(x, t) representan soluciones de
una ecuación llamada ecuación de onda lineal. Esta es:

\begin{equation*}
	\dfrac{\partial^{2} y}{\partial x^2} = \dfrac{1}{v^2} \dfrac{\partial^2 y}{\partial t^2}
\end{equation*}

\section{Ondas sonoras}
Las \textbf{sonoras} viajan a través de \textbf{cualquier } medio material con una rapidez que depende
de las \textbf{propiedades del medio}.\\
\linebreak
A medida que las ondas sonoras viajan a través del aire, los
elementos del aire \textbf{vibran} para producir cambios en \textbf{densidad} y \textbf{presión} a lo largo de la
dirección del movimiento de la onda.\\ 
\linebreak
Si la fuente de las ondas sonoras vibra sinusoidalmente, las variaciones de presión también son sinusoidales.\\
\linebreak
Las ondas sonoras se dividen en \textbf{tres categorías} que cubren diferentes intervalos de frecuencia.\\ 
\begin{enumerate}
	\item Las \textbf{ondas audibles} se encuentran dentro del intervalo de sensibilidad del oído
		humano.
	\item Las \textbf{ondas infrasónicas} tienen frecuencias por abajo del
		intervalo audible.
	\item Las \textbf{ondas ultrasónicas} tienen frecuencias por arriba del alcance audible.
\end{enumerate}

\paragraph{Rapidez de ondas sonoras}

Un pistón en el extremo izquierdo se mueve hacia la derecha para \textbf{comprimir el gas} y crear el pulso.
Antes de que el pistón se mueva, el gas \textbf{no está perturbado} y tiene \textbf{densidad uniforme},
cuando el pistón se empuja súbitamente hacia la derecha, el gas justo \textbf{enfrente} de él se \textbf{comprime};
la presión y la densidad en esta región ahora son mayores de lo que eran antes de que el
pistón se moviera. Cuando el pistón \textbf{se detiene}, la región comprimida del
gas \textbf{continúa en movimiento} hacia la derecha, lo que corresponde a un \textbf{pulso longitudinal}
que viaja a través del tubo con rapidez $v$.\\
\linebreak
La \textbf{rapidez} de las ondas sonoras en un medio depende de la \textbf{compresibilidad} y la \textbf{densidad} del medio; si éste es un \textbf{líquido} o un \textbf{gas} y tiene un módulo volumétrico B y densidad $p$, la rapidez de las ondas sonoras en dicho medio es

\begin{equation*}
	v = \sqrt[2]{\dfrac{B}{p}}
\end{equation*}

La rapidez del sonido también depende de la \textbf{temperatura} del medio. La relación entre
la rapidez de la onda y la temperatura del aire, para sonido que viaja a través del aire, es:

\begin{equation*}
	v = (331 m/s) \sqrt{1+\frac{T_c}{273ºC}}
\end{equation*}

Donde 331 m/s es la rapidez del sonido en aire a 0°C y $T_c$ es la \textbf{temperatura} del aire en grados celsius.

\paragraph{Ondas sonoras periódicas}
Una región comprimida se forma siempre que el pistón se empuje
en el tubo. Esta región comprimida, llamada \textbf{compresión}, se mueve a través del tubo, y
comprime continuamente la región justo enfrente de ella misma.\\
\linebreak
Cuando el pistón se jala hacia atrás, el gas enfrente de él se expande y la presión y la densidad en esta región caen
por abajo de sus valores de equilibrio. Estas regiones de baja presión, llamadas \textbf{enrarecimiento},
también se propagan a lo largo del tubo, siguiendo las compresiones.\\
A medida que el pistón tiene una oscilación sinusoidal, se establecen continuamente
regiones de compresión y enrarecimiento.\\
\linebreak 
La distancia entre dos compresiones sucesivas (o dos enrarecimientos sucesivos) iguala 
la \textbf{longitud de onda} $\lambda$ de la onda sonora.\\
\linebreak
Mientras estas regiones viajan a través del tubo, cualquier elemento pequeño del medio se mueve
con movimiento \textbf{armónico simple paralelo} a la dirección de la onda.\\
\linebreak
Si $s(x, t)$ es la posición de un elemento pequeño en relación con su posición de equilibrio,
 se puede expresar esta función de posición armónica como:
\begin{equation*}
 	s(x,t) = s_{max} \cos(kx-wt)
\end{equation*}

Donde $s_{max}$ es la posicion maxima del elemento relativo al equilibrio y se denomina \textbf{Amplitud de desplazamiento}.
( el desplazamiento del elemento es a lo largo de $x$)

La variación en la presión del gas $\Delta P$ observada desde el valor de equilibrio también es
periódica. $\Delta P$ se conoce por:

\begin{equation*}
 	\Delta p(x,t) = \Delta p_{max} \cos(kx-wt)
\end{equation*}

donde la amplitud de presión $\Delta p_{max}$, que es el \textbf{cambio máximo en presión} desde el valor
de equilibrio, se proporciona por

\begin{equation*}
	\Delta p_{max} = pvws_{max}
\end{equation*}

La variación de presión es un \textbf{máximo} cuando el desplazamiento
desde el equilibrio es \textbf{cero}, y el desplazamiento desde el equilibrio es un \textbf{máximo} cuando
la variación de presión es \textbf{cero}.

\paragraph{Intensidad de ondas sonoras periódicas}

se demostró que una onda que viaja sobre una cuerda tensa transporta
energía. Se aplica el mismo concepto a ondas sonoras. La energía cinética en una longitud de onda de
la onda sonora es

\begin{equation*}
	K_{\lambda} = \frac{1}{4} (pA)w^2s^2_{max}\lambda	
\end{equation*}

la \textbf{energía mecánica} total para una longitud de onda es

\begin{equation*}
	E_{\lambda} = K_{\lambda} + U_{\lambda}  = \frac{1}{2} (pA)w^2s^2_{max}\lambda
\end{equation*}

la rapidez de transferencia de energía es

\begin{equation*}
	P = \frac{1}{2} pAvw^2s^2_{max}\lambda
\end{equation*}

donde $v$ es la rapidez del sonido en el aire.

La \textbf{intensidad} $I$ de una onda, se define como la
\textbf{rapidez} a la cual la energía transportada por la onda se transfiere a través de una unidad
de área A \textbf{perpendicular} a la dirección de viaje de la onda:

\begin{equation*}
	I = \dfrac{P}{A}
\end{equation*}

\begin{equation*}
	I = \dfrac{1}{2} pv(ws_{max})^2
\end{equation*}

\begin{equation*}
	I = \dfrac{(\Delta P_{max})^2}{2pv}
\end{equation*}

Cuando una fuente emite sonido por igual en todas
direcciones, el resultado es una \textbf{onda esférica}.\\
Cada arco representa
una superficie sobre la cual es constante la fase de la onda. A tal superficie 
de fase constante se le llama \textbf{frente de onda}.\\
Las líneas radiales que se dirigen hacia
afuera desde la fuente se llaman \textbf{rayos}.\\
\linebreak
La potencia promedio $P_{prom}$ emitida por la fuente debe tener una distribución uniforme sobre cada frente de onda esférica de área. Por tanto, la intensidad de la onda a
una distancia $r$ de la fuente es

\begin{equation*}
	I = \dfrac{P_{prom}}{4\pi r^2}
\end{equation*}

\paragraph{Nivel sonoro en decibeles}

El nivel sonoro $\beta$ se define mediante la ecuación

\begin{equation*}
	\beta = 10 \log \left( \dfrac{I}{I_0} \right)
\end{equation*}

La constante $I_0$ es la \textbf{intensidad de referencia}, considerada como el umbral de audición
($I_0 = 1 * 10^{-12}$ W/$m^2$) $I$ es la intensidad en watts por cada metro cuadrado a la 
que corresponde el nivel de sonido $\beta$, donde $\beta$ se mide en decibeles (dB).

\paragraph{El efecto Doppler}

Tal vez haya notado cómo \textbf{varía el sonido} del claxon de un vehículo a medida que éste
se aleja. La \textbf{frecuencia} del sonido que escucha mientras el vehículo se aproxima a usted es
\textbf{más alta} que la frecuencia que escucha mientras se aleja. Esta experiencia es un ejemplo
del \textbf{efecto Doppler}.\\
\linebreak

Sean $f$ la frecuencia de la fuente, $\lambda$ la longitud de onda y $v$ la rapidez del sonido en
la figura Si el observador también queda estable, detectará frentes de onda a una frecuencia $f$.
Cuando el observador se mueve \textbf{hacia} la fuente, la rapidez de las ondas
relativa al observador es $v' = v + v_0$, pero la
longitud de onda $\lambda$ no cambia. Se puede decir
que la frecuencia $f$ que escucha el observador está \textbf{aumentada} y se conoce por
\begin{equation*}
	f' = \dfrac{v'}{\lambda} = \dfrac{v+v_0}{\lambda}
\end{equation*}
Ya que $\lambda = v/f$, $f'$ se puede expresar como:
\begin{equation*}
	f' = \left( \dfrac{v+v_0}{\lambda}  \right)f
\end{equation*}

Si el observador es móvil alejándose de la fuente, la rapidez de la onda 
relativa al observador es $v' = v-v_0$. En este caso la frecuencia escuchada por el observador queda \textbf{reducida}
y se encuentra por:

\begin{equation*}
	f' = \left( \dfrac{v-v_0}{\lambda}  \right)f
\end{equation*}


Ahora suponga que la \textbf{fuente} está en movimiento y que el observador queda en reposo.\\
Si la fuente avanza directo hacia el observador, los \textbf{frentes de onda} escuchados por el observador están más juntos de lo que estarían si la fuente no se moviera.


Como resultado, la longitud de onda $\lambda$ medida por el observador es mas corta que la
longitud de onda $\lambda$ de la fuente. Durante cada vibración, que dura un intervalo de tiempo
$T$ (el periodo), la fuente se mueve una distancia $v_sT = \frac{v_s}{f}$ y la longitud de onda se \textbf{acorta}
en esta cantidad. Por lo tanto, la longitud de onda observada $\lambda'$ es:

\begin{equation*}
	\lambda' = \lambda - \Delta \lambda = \lambda - \dfrac{v_s}{f}
\end{equation*}


Como $\lambda = v/f$, la frecuencia $f$ que escucha el observador es:

\begin{equation*}
	f' = \dfrac{v}{(v/f)-(v_s/f)}	
\end{equation*}

La frecuencia observada aumenta siempre que la fuente se mueva hacia el observador.

\begin{equation*}
	f' = \left( \dfrac{v}{v-v_s}\right)f
\end{equation*}
La frecuencia observada disminuye siempre que la fuente se aleje del observador:
\begin{equation*}
	f' = \left( \dfrac{v}{v+v_s}\right)f
\end{equation*}

Por último, al combinar las ecuaciones se obtiene la siguiente correspondencia general para la frecuencia observada:

\begin{equation*}
	f' = \left( \dfrac{v \pm v_0}{v \pm v_s}\right)f
\end{equation*}

\end{document}