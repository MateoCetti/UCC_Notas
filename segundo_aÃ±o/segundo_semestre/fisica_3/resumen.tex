\title{Resumen de Fisica 3}

\author{Mateo P. Cetti}

\documentclass[10pt]{article}

\usepackage{amsmath}
\usepackage{amsfonts}

\begin{document} 
\maketitle

\section{Movimiento ondulatorio}

El mundo está lleno de \textbf{ondas}, los dos tipos principales son las ondas \textbf{mecánicas} 
(Tirar una piedra al agua) y las ondas \textbf{electromagnéticas} (las ondas de radio). En el caso de las 
ondas mecánicas, algunos medios físicos se perturban. Las ondas
electromagnéticas no requieren un medio para propagarse.\\
La caracteristica principal del movimiento ondulatorio es que la \textbf{energía} se 
transfiere a través de una distancia, perola \textbf{materia} no.

\paragraph{Ondas mecanicas}

Todas las ondas mecánicas requieren:

\begin{enumerate}
    \item alguna fuente de perturbación
    \item un medio que contenga elementos que sean factibles de perturbación
    \item algún mecanismo físico a partir del cual los elementos del medio puedan influirse mutuamente.
\end{enumerate}

\paragraph{Tipos de ondas: }

\begin{itemize}
    \item \textbf{onda transversal} Una onda viajera o pulso que hace que los elementos 
    del medio perturbado se muevan \textbf{perpendiculares} a la dirección de propagación.
    \item \textbf{onda longitudinal} Una onda viajera o pulso que mueve a los elementos del medio en \textbf{paralelo}
    a la dirección de propagación
\end{itemize}

\paragraph{Funcion de onda}
\begin{itemize}
    \item $y(x, t) y(x-vt, 0)$ (Si el pulso viaja hacia la \textbf{derecha} en X)
    \item $y(x, t) y(x+vt, 0)$ (Si el pulso viaja hacia la \textbf{izquierda} en X)
\end{itemize}

La función de onda y(x, t) representa la coordenada y, la posición transversal,
de cualquier \textbf{elemento} ubicado en la posición x en cualquier tiempo t.\\
\linebreak
Ademas, si t es fijo (como en el caso de tomar una instantánea del pulso), la función 
de onda y(x), a veces llamada \textbf{forma de onda}, define una curva que representa la 
\textbf{forma} geométrica del \textbf{pulso} en dicho tiempo.

\paragraph{El modelo de onda progresiva}

Primero, la forma de onda completa se mueve hacia la derecha de modo que la curva 
se mueve hacia la derecha y al final llega a la posición de la curva azul. Este es el \textbf{movimiento 
de la onda}. Si se concentra en un elemento del medio, como el elemento en x = 0, observará que cada
elemento se mueve hacia arriba y hacia abajo a lo largo del eje y en movimiento armónico
simple. Este es el \textbf{movimiento de los elementos del medio}. Es importante \textbf{diferenciar} entre el
movimiento de la onda y el movimiento de los elementos del medio.

\paragraph{modelo de onda}

Un punto en la figura en que el desplazamiento del elemento de su 
posición normal está más alto se llama \textbf{cresta} de la onda.\\

El punto más bajo se llama \textbf{valle}.\\

la longitud de onda ($\lambda$) es la \textbf{distancia mínima} entre dos puntos
cualesquiera en ondas adyacentes\\

\textbf{periodo} T de las ondas es el intervalo de tiempo requerido 
para que dos puntos idénticos de ondas adyacentes pasen por un punto.\\

\textbf{frecuencia} $f$ de una onda periódica es el número de crestas (o
valles o cualquier otro punto en la onda) que pasa un punto determinado en un intervalo
de tiempo \textbf{unitario}.\\

\begin{equation*}
    f = \dfrac{1}{T}
\end{equation*}

La \textbf{máxima posición} de un elemento del medio relativo a su \textbf{posición de equilibrio} se
llama \textbf{amplitud} A de la onda.\\

Las ondas viajan con una \textbf{rapidez} específica, y esta rapidez depende de las \textbf{propiedades
del medio perturbado}.\\

\paragraph{Formas de la funcion de onda:}
Forma 1:
\begin{equation*}
    y(x, t) =  A sen \left[ \frac{2\pi}{\lambda} (x-vt) \right] 
\end{equation*}

$v = \dfrac{\lambda}{T}$. Sustituyendo esto en la ecuacion anterior:

\begin{equation*}
    y = Asen \left[ 2\pi \left( \frac{x}{\lambda} - \frac{t}{T} \right) \right] 
\end{equation*}

Para abreviar la formula definimos:\\
\linebreak

\textbf{número de onda angular: }$k = \dfrac{2\pi}{\lambda}$

\textbf{frecuencia angular: }$w = \dfrac{2\pi}{T} = 2\pi f$\\
\linebreak
Entonces la funcion queda:

\begin{equation*}
    y = Asen(kx-wt)
\end{equation*}

Ademas, $v$ se puede definir alternativamente como: $v = \frac{w}{f} = \lambda f$\\
\linebreak
La función de onda conocida en la ecuación supone que la posición vertical y de
un elemento del medio es cero en $x=0$ y $t=0$. Este no necesita ser el caso. Si no lo es, la
función de onda por lo general se expresa en la forma:\\
$y = Asen(kx-wt + \phi)$\\
donde $\phi$ es la \textbf{constante de fase}

\paragraph{Ondas sinusoiodales en cuerdas}

Se puede usar esta expresión para describir el movimiento de cualquier elemento de la
cuerda. Un elemento en el punto P se mueve sólo verticalmente, y de este modo su coordenada x 
permanece constante. Por lo tanto, la rapidez transversal $v_y$ (no confundir con la 
rapidez de onda v) y la aceleración transversal $a_y$ de los \textbf{elementos} de la cuerda son

\begin{equation*}
    v_y = -wA cos(kx-wt)
\end{equation*}

\begin{equation*}
    a_y = -w^2 A sen(kx-wt)
\end{equation*}

Los valores \textbf{máximos} de la rapidez transversal y la
aceleración transversal son simplemente los valores absolutos de los coeficientes de las
funciones coseno y seno:

\begin{equation*}
    v_y max = wA
\end{equation*}

\begin{equation*}
    a_y max = w^2 A
\end{equation*}

La \textbf{rapidez transversal} y la \textbf{aceleración transversal} de los elementos de la cuerda no llegan
\textbf{simultáneamente} a sus valores máximos. La rapidez transversal llega a su valor máximo
($wA$) cuando $y=0$, mientras que la magnitud de la aceleración transversal llega a su valor
máximo ($w^2A$) cuando $y= \pm A$

\paragraph{La rapidez de ondas en cuerdas}

En esta sección se determina la rapidez de un pulso transversal que viaja en una cuerda
tensa. Primero cabe mencionar que se espera que la \textbf{rapidez} de la onda \textbf{aumente} con una
\textbf{tensión creciente} y debe disminuir a medida que \textbf{aumente} la \textbf{masa por unidad de longitud} 
de la cuerda.\\
\linebreak
Si la tensión en la cuerda es $T$ y su masa por unidad
de longitud es $\mu$, la rapidez de onda es:

\begin{equation*}
    \sqrt{\dfrac{T}{\mu}}
\end{equation*}

\paragraph{Reflexión y transmisión}

El modelo de onda progresiva describe ondas que viajan a través de un medio uniforme sin
interactuar con algo más en el camino. Ahora se considerará cómo una onda progresiva
es afectada cuando \textbf{encuentra un cambio} en el medio\\
\linebreak
considere un pulso que viaja en una cuerda que está rígidamente unida a un soporte en un extremo, como en
la figura. Cuando el pulso alcanza el soporte, se presenta un cambio severo en el medio: la
cuerda termina. Como resultado, el pulso experimenta reflexión; es decir, el pulso se
mueve de regreso a lo largo de la cuerda en la dirección opuesta.Note que el pulso reflejado está invertido.\\
\linebreak
Ahora considere otro caso. Esta vez, el pulso llega al final de una cuerda que es libre
de moverse verticalmente, como en la figura 16.14. La tensión en el extremo libre se mantiene porque la cuerda está amarrada a un anillo de masa despreciable que tiene libertad
para deslizarse verticalmente sobre un poste uniforme sin fricción. De nuevo, el pulso
se refleja, pero esta vez no se invierte. \\
\linebreak
Para finalizar, considere una situación en la que la frontera es intermedia entre estos
dos extremos. En este caso, parte de la energía en el pulso incidente se refleja y parte se
somete a transmisión; es decir: parte de la energía pasa a través de la frontera. Por ejemplo, 
suponga que una cuerda ligera se une a una cuerda más pesada,

\paragraph{Rapidez de transferencia de energía mediante ondas sinusoidales
en cuerdas}

La energía cinética de un elemento de la cuerda se expresa como: 
\begin{equation*}
    dK = \frac{1}{2} (\mu dx) v_y^2
\end{equation*}

Al sustituir con la ecuación para la rapidez transversal general 
de un oscilador armónico simple se obtiene:

\begin{equation*}
    dK = \frac{1}{2} \mu w^2 A^2 cos^2 (kx-wt) dx 
\end{equation*}

Si se toma una instantánea de la onda en el tiempo t=0, la energía cinética de un elemento dado es:

\begin{equation*}
    dK = \frac{1}{2} \mu w^2 A^2 cos^2 (kx) dx 
\end{equation*}

Al integrar esta expresión sobre todos los elementos de cuerda en una longitud de onda
de la onda produce la energía cinética total KA en una longitud de onda:

\begin{equation*}
    dK_\lambda = \frac{1}{4} \mu w^2 A^2 \lambda 
\end{equation*}

para la energía potencial total $U_A$
en una longitud de onda produce exactamente el mismo resultado:

\begin{equation*}
    dU_\lambda = \frac{1}{4} \mu w^2 A^2 \lambda 
\end{equation*}

La energía total en una longitud de onda de la onda es la suma de las energías potencial
y cinética:

\begin{equation*}
    E_\lambda = U_\lambda + K_\lambda =  \frac{1}{4} \mu w^2 A^2 \lambda 
\end{equation*}

Por lo tanto, la potencia $\mathbb{P}$, o rapidez de transferencia de energía $T_{OM}$ asociada
con la onda mecánica, es

\begin{equation*}
    \mathbb{P} = \frac{1}{2} \mu w^2 A^2 v
\end{equation*}

\paragraph{La ecuación de onda lineal}

Todas las funciones de onda y(x, t) representan soluciones de
una ecuación llamada ecuación de onda lineal.

\end{document}