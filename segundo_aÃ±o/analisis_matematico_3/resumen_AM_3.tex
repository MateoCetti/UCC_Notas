\title{Resumen de analisis matematico III}
\author{
        Mateo P. Cetti \\
        Estudiante - Universdad Catolica de Cordoba\\
        Ing Ambrosio Taravella, 6240, \underline{Cordoba, Argentina}
}
\date{\today}

\documentclass[10pt]{article}

\usepackage{amsmath}
\usepackage{amsfonts}

\begin{document} 
\maketitle

\section{Introduccion [03/08/20]}
En esta meteria vamos a ver funciones como en AM2 solo que esta vez la entrada y salida esta acompañada de \textbf{numeros complejos}
\begin{itemize}
	\item \textbf{Libro:} Variable compleja y sus aplicaciones (Churchil algo)
\end{itemize}

\section{Numeros complejos}
El conjunto $\mathbb C$ de los numeros complejos esta dado como:
\begin{equation*}
	\left\lbrace \mathbb{C}=a+bi   |    a,b    \epsilon \mathbb{R} y i = -1\right\rbrace
\end{equation*}

El simbolo $i$ con la propiedad de que $i^2 = -1$ se denomina unidad \textbf{imaginaria}

Suma y multiplicacion en $\mathbb{C}$:
\begin{itemize}
	\item \textbf{Suma} $(a+bi) + (c+di) = (a+c) + (b+d)i$
	\item \textbf{Multiplicacion} $(a+bi)(c+di) = (ac- bd) + (ad+bc)i$ (menos porque queda $i^2$)
\end{itemize}

Con estas operaciones, el conjunto $\mathbb{C}$ es un \textbf{Conjunto algebraico.}\\

si $z = a+bi$

\begin{itemize}
	\item \textbf{Parte real de z:} a (Notacion: Re[z])
	\item \textbf{Parte imaginaria de z:} b (Notacion: Im[z])
\end{itemize}

\paragraph{Conjugado de un numero complejo (z)}
Dado el numero complejo $z = a+bi$, el numero complejo $x- bi$ se denomina conjugado de $z$ y se denota $\overline{z}$\\

Dado el numero complejo $z = a+bi$, le podemos asignar el par ordenado $(x,y)$, y reciprocamente, dado el par ordenado $(x,y)$, le podemos asignar el numero complejo $z = a+bi$, de modo que existe una \textbf{relacion biunivoca} entre $\mathbb{C}$ y $\mathbb{R}^2$\\

Mirando el plano cartesiano, el eje horizontal se denomina \textbf{eje real} y el eje vertical se denomina \textbf{eje imaginario}. Este plano se denomina \textbf{plano complejo}.\\

\paragraph{Observacion: }Si $a+bi$ = $c+di$ entonces $a=c$ y $b=d$

\paragraph{Modulo de z: } Dado el numero complejo $z = a+bi$, el modulo de $z$ que denotamos $|z|$ esta dado por:

\begin{equation*}
	|z| = \sqrt[2]{a^2+b^2}
\end{equation*}

\begin{center}
	\textbf{Propiedades}
\end{center}
\begin{enumerate}
	\item $|z_1 z_2| = |z_1||z_2|$
	\item $\left|\dfrac{z_1}{z_2}\right| = \dfrac{|z_1|}{|z_2|}$
	\item $|z_1 + z_2| \leq |z_1|+|z_2|$
	\item $z\overline{z} = |z|^2$
	\item $|z_1 - z_2| \geq |z_1|-|z_2|$ 
\end{enumerate}

\begin{center}
\textbf{Una aplicacion de las propiedades de los $\mathbb{C}$}
\end{center}

\begin{equation*}
	\dfrac{1}{a+bi} = \dfrac{a-bi}{(a+bi)(a-bi} = \dfrac{a-bi}{a^2+b^2} = \dfrac{a}{a^2+b^2} + \dfrac{bi}{a^2+b^2}
\end{equation*}
\begin{equation*}
	Re \left[\frac{1}{z}\right] = \dfrac{a}{|z|^2}	
\end{equation*}
\begin{equation*}
	Im \left[\frac{1}{z}\right] = \dfrac{bi}{|z|^2}
\end{equation*}

\paragraph{Forma polar o trigonometrica de un $\mathbb{C}$}

\begin{itemize}
	\item $a = |z|.cos \theta$
	\item $b = |z|.sen \theta$
\end{itemize}

\begin{equation*}
	z = |z|.cos\theta + |z|.i.sen\theta = |z|.(cos\theta+isen\theta) 
\end{equation*}

El Angulo $\theta$ se denomina \textbf{argumento} del numero complejo.\\

\paragraph{Formula de Moivre: }

Si $z = |z|.(cos\theta+isen\theta$ entonces:

\begin{equation*}
	z^n = |z|^n.(cos(n\theta)+isen(n\theta))
\end{equation*}

\paragraph{Funciones complejas}

Sea $f:\mathbb{C}\rightarrow\mathbb{C}$ dada por:

\begin{equation*}
	f(z) = z^2
\end{equation*}

Esto es,

\begin{equation*}
	f(a+bi) = (a+bi)^2 = a^2+b^2+2abi
\end{equation*}

En general:

\begin{center}
	$f(a+bi) = u(a,b)+ iv(a,b)$ Donde $u,v: \mathbb{R}^2 \rightarrow \mathbb{R}$ 
\end{center}

\paragraph{Limite y Continuidad} 
\paragraph{Limite de numeros complejos: }sean $f: \mathbb{C} \rightarrow \mathbb{C}$ y $z_0$ un punto de acumulacion
del dominio de $f$. El numero complejo L es el limite de $f$ en $z_0$ si para todo $\epsilon > 0$ existe
$\delta > 0$, tal que $|f(z)-L| < \epsilon$ cuando $z \in dom(f)$ y $0 < |z-z_0|< \delta$

\paragraph{Continuidad de numeros complejos: }sean $f: \mathbb{C} \rightarrow \mathbb{C}$ y $z_0$ un punto
del dominio de $f$, diremos que $f$ es continua en $z_0$ si para todo $\epsilon > 0$ existe
$\delta > 0$, tal que $|f(z)- f(z_0)| < \epsilon$ cuando $z \in dom(f)$ y $0 < |z-z_0|< \delta$.\\
\begin{equation*}
	\left( \lim_{z \rightarrow z_0} f(z) = f(z_0) \right)
\end{equation*}

\section{Derivada y exponencial compleja}

sean $f: \mathbb{C} \rightarrow \mathbb{C}$ y $z_0 \in dom(f)$. Diremos que $f$ es derivable en $z_0$
si existe $\lim_{z \rightarrow z_0} \dfrac{f(z)-f(z_0)}{z-z_0}$.\\
En este caso se llama a dicho limite \textbf{derivada} de $f$ en $z_0$ y la denotamos $f'(z_0)$.

\paragraph{Ecuaciones de Cauchy - Riemann} (Anda que te las demuestro)

\begin{equation*}
	\dfrac{\partial u}{\partial x} (x_0, y_0) = \dfrac{\partial v}{\partial y} (x_0, y_0)
\end{equation*}

\begin{equation*}
	\dfrac{\partial u}{\partial y} (x_0, y_0) = -\dfrac{\partial v}{\partial x} (x_0, y_0)
\end{equation*}

Sea $f: u+iv$ tal que $u, v: \mathbb{R}^2 \rightarrow \mathbb{R}$ son de clase $C^1$. si $\dfrac{\partial u}{\partial x} (x_0, y_0) = \dfrac{\partial v}{\partial y} (x_0, y_0)$ y $\dfrac{\partial u}{\partial y} (x_0, y_0) = -\dfrac{\partial v}{\partial x} (x_0, y_0)$ en algun conjunto A $\subset dom(u) \cap dom(v)$ entonces $f$ es derivable en el conjunto A.\\
\linebreak

\paragraph{funcion analitica}Sean $f: \mathbb{C} \rightarrow \mathbb{C}$ y $z_0 \in dom(f)$. Diremos que $f$ es \textbf{analitica} en el punto $z_0$ si existe $r > 0$ tal que $B_r(z_0) \subset dom(f)$ y $f$ es derivable en todo punto de $B_r(z_0)$\\
\linebreak

Sean $f: \mathbb{C} \rightarrow \mathbb{C}$ y $z_0 \in dom(f)$. Si $f$ es analitica en $B_r(z_0) - {z_0}$ e dice que $z_0$ es un punto singular o \textbf{Singularidad} de $f$

\paragraph{Exponencial compleja}
Las siguientes condiciones son validas tambien para los numeros complejos:
\begin{enumerate}
	\item $e^0 = 1$
	\item $e^a e^b = e^{a+b} $
	\item $\dfrac{de^x}{dx} = e^x$
\end{enumerate}

\begin{equation*}
	e^{x+yi} = e^{(x+0i)+(0+yi)} = e^{x+0i}e^{0+yi} = e^{x}e^{yi}
\end{equation*}

\paragraph{Formula de Eeuler} (Sin demostracion tambien)
\begin{equation*}
	e^{iy} = cos(y)+isin(y)
\end{equation*}

\paragraph{Propiedades basicas de la exponencial compleja:}

\begin{enumerate}
	\item $\overline{e^z} =e^{\overline{z}}$
	\item $e^z$ es \textbf{analitica} en todo el plano complejo
	\item $e^z \neq 0$ para todo $z \in \mathbb{C}$
	\item $|e^z| = e^{Re(z)}$
\end{enumerate}

\section{Clase 3 [24/08/20]}

\paragraph{Logaritmo complejo} 
Sea $g: \mathbb{C}\rightarrow \mathbb{C}$ analitica en algun subconjunto abierto y conexo de $\mathbb{C}$ tal que:
\begin{equation*}
	e^{g(z)} = z
\end{equation*}
Si $g(z) = Re[g(z)] + i Im[g(z)]$, entonces
\begin{equation*}
	e^{g(z)} = z \Rightarrow e^{Re[g(z)] + i Im[g(z)]} = |z|e^{iArg(z)}
\end{equation*}
De donde
\begin{itemize}
	\item $Re[g(z)] = ln|z|$ ($e^{Re[g(z)]} = |z|$)
	\item $Im[g(z)] = Arg[g(z)] + 2k\pi$
\end{itemize}
La funcion $g$ se denomina logaritmo complejo de y lo denotamos $log$, esto es
\begin{equation*}
	log(z) = ln|z| + iArg(z)+2k\pi
\end{equation*}
En el caso que $k=0$, este logaritmo se denomina \textbf{rama principal} 
de la funcion logaritmo y la denominamos $Log$, esto es
\begin{equation*}
	Log(z) = ln|z| + iArg(z)
\end{equation*}
\paragraph{Algunas propiedades}
Como
\begin{equation*}
	e^{log(z)} = z
\end{equation*}
Derivando ambos miembros tenemos
\begin{equation*}
	e^{log(z)} \cdot log'(z) = 1
\end{equation*}
De donde
\begin{equation*}
	log'(z) =\dfrac{1}{e^{log(z)}} = \dfrac{1}{z}
\end{equation*}
Sean $z_1$ y $z_2$ numeros complejos con $z_1 \neq 0$
\begin{equation*}
	z_1^{z_2} = e^{z_2 log(z_1)}
\end{equation*}
\paragraph{Funciones trigonometricas complejas}
Sea $z \in \mathbb{C}$, entonces el seno y coseno de $z$ estan dados por:
\begin{equation*}
	cos(z) = \dfrac{1}{2}(e^{iy}+e^{-iy})
\end{equation*}
\begin{equation*}
	sin(z) = \dfrac{1}{2}(e^{iy}-e^{-iy})
\end{equation*}
\paragraph{Funciones Armonicas}
Una funcion $u:\mathbb{R}^2 \rightarrow \mathbb{R}$ que satisface la \textbf{ecuacion de Laplace} se denomina funcion armonica
\paragraph{Funcion de Laplace}
\begin{equation*}
	\dfrac{\partial^2 u}{\partial^2 x} + \dfrac{\partial^2 u}{\partial^2 y}  =0
\end{equation*}

Luego, si $f = u+iv$ es una \textbf{funcion analitica} $u$ y $v$ son \textbf{funciones armonicas} y se dice que $v$ es \textbf{armonica conjugada} de $u$

\paragraph{Integrales de linea en el plano complejo}

Sea $\gamma : [a,b] \rightarrow \mathbb{C}$ tal que $\gamma (t) = \gamma_1 (t) + i\gamma_2 (t)$ entonces:
\begin{equation*}
	\int_a^b \gamma (t)dt = \int_a^b \gamma_1 (t)dt + i\int_a^b \gamma_2 (t) dt
\end{equation*}
Sea $f: \mathbb{C}\rightarrow \mathbb{C}$ continua y sea $C$ una curva de clase $C^1$ contenida en $dom(f)$. Si $C$ esta parametrizada por $\gamma : [a,b] \rightarrow \mathbb{C}$, entonces la integral de linea de $f$ a lo largo de $c$, que denotamos $\int_c f$ esta dada por
\begin{equation*}
	\int_c f = \int_a^b f(\gamma(t))\gamma'(t)dt
\end{equation*}
\paragraph{Propiedades de la integral de linea compleja}
\begin{enumerate}
	\item $\int_c af = a\int_c f$
	\item $int_c (f+g) = \int_c f + \int_c g$
	\item $|\int_c f| \leq \int_c |f|$
	\item Si $C_1$ y $C_2$ son curvas tales que $C_1 \bigcup C_2$ esta contenida den $Dom(f)$, entonces
	$\int_{C_1 \bigcup C_2} f = \int_{c_1}f + \int_{c_2}f - \int_{C_1 \bigcap C_2}$
\end{enumerate}

\section{•}

\paragraph{Notacion} si $\Gamma$ es una curva cerrada simple, con $\Gamma^0$ denotaremos a la \textbf{region acotada} por la curva

\paragraph{Teorema de Cauchy - Goursat} Sea $f: \mathbb{C} \rightarrow \mathbb{C}$ y $C$ una curva cerrada simple de clase $C^1$ contenida en el dominio de $f$ tal que $f$ es analitica sobre $\Gamma^0 \cup \Gamma$. Entonces
\begin{equation*}
	\int_{\Gamma} f = 0
\end{equation*}


\paragraph{Demostracion} Supongamos que $f = u+iv$ y sea $\gamma : [a,b] \rightarrow \mathbb{C}$ una parametrizacion de $\Gamma$. Si $\gamma(t) = \gamma_1(t) + i\gamma_2 (t)$ entonces\\

\begin{equation*}
	\int_{\Gamma} f = \int_a^b f(\gamma (t))\gamma' (t)dt = \int_a^b [u(\gamma_1(t), \gamma_2(t)) +i v(\gamma_1(t), \gamma_2(t))] [\gamma'_1(t) + i\gamma'_2(t)]dt
\end{equation*}

\begin{equation*}
	\int_{\Gamma} f = \int_a^b u(\gamma_1(t), \gamma_2(t))\gamma'_1(t)) +i v(\gamma_1(t), \gamma_2(t))\gamma'_2(t)) dt + 
	i\int_a^b [v(\gamma_1(t), \gamma_2(t))\gamma'_1(t)) + u(\gamma_1(t), \gamma_2(t))\gamma'_2(t))]
\end{equation*}

\begin{equation*}
	\int_a^b u(\gamma(t)),-v(\gamma(t))\cdot \gamma'(t)dt + i\int_a^b (v(\gamma(t)), u(\gamma(t) )\cdot\gamma'(t)dt
\end{equation*}

Por el teorema de green

\begin{equation*}
	\int_{\Gamma} f = \int_{\Gamma^0}(- \dfrac{\partial v}{\partial x}(x, y) -\dfrac{\partial u}{\partial y} (x,y) )dxdy +i \int_{\Gamma^0} (\dfrac{\partial u}{\partial x} (x,y) - \dfrac{\partial v}{\partial y}(x,y))dxdy
\end{equation*}

y como $f$ es analitica

\begin{equation*}
- \dfrac{\partial v}{\partial x}(x, y) -\dfrac{\partial u}{\partial y} (x,y)  = 0
\end{equation*}
\begin{equation*}
\dfrac{\partial u}{\partial x} (x,y) - \dfrac{\partial v}{\partial y}(x,y) = 0
\end{equation*}	

Por lo tanto $\int_{\Gamma} f = 0$	

\paragraph{Definicion} Sea $f: \mathbb{C} \rightarrow \mathbb{C}$ y $A$ un conjunto contenido en $dom(f)$. Si $z_0$ es un punto de $A$ y $f$ es analitica en el conjunta $A - {z_0}$ se dice que $z_0$ es un punto \textbf{singular} de $f$

\paragraph{Teorema de Cauchy - Goursat generalizado} Sea $f: \mathbb{C} \rightarrow \mathbb{C}$ y $C$ una curva cerrada simple de clase $C^1$ y sean $z_1, z-2,\dots, z_n \in C$ tal que $f$ es analitica en la region $[C^0 \cup C] - {z_1,\dots, z_n}$. Si $r_1,\dots ,r_n$ son numeros reales positivostales que $C_{r_j}(z_j) \cap C_{r_k}(z_k) = \emptyset$ si $j \neq k$

\begin{equation}
	\int_c f = \sum_{j=1}
\end{equation}

\paragraph{Observacion} Si $C_1$ y $C_2$ son curvas cerradas simples de clase $C^1$ tales que $C_1\subset C_2^0$ entonces

\begin{equation*}
	\int_{c_1} f = \int_{c_2} f
\end{equation*}
Se conoce a esto como el \textbf{principio de deformacion de contorno}

\paragraph{Teorema / Formula de la integral de Cauchy} Sea $f: \mathbb{C} \rightarrow \mathbb{C}$ analitica y $C$ una curva cerrada simple de clase $C^1$ contenida en el $Dom(f)$. Si $z_0 \in C^0$, entonces 

\begin{equation*}
	f(z_0) = \dfrac{1}{2\pi i} \cdot \int_c \dfrac{f(z)}{z-z_0}dz
\end{equation*}

\section{series y sucesiones}

\paragraph{Suceciones} Una sucesion  en $\mathbb{R}$ es una funcion $\phi : \mathbb{N}\rightarrow \mathbb{R}$. Si el valor de $\phi$ en un natural $n$ es $S_n$ ($\phi(n) = S_n$) denotamos a esta sucesion como ${S_n}$\\
\linebreak
Ejemplo
\begin{itemize}
	\item $S_n = \dfrac{1}{(1+i)^n}$ 
\end{itemize} 	

Dada la sucesion ${S_n}$ en $\mathbb{C}$, si existe el $\liminf_{n\rightarrow \infty} S_n$, diremos que la sucesion es convergente, y si$\liminf_{n\rightarrow \infty} S_n = z_0$ decimos que ${S_n}$ converge en $z_0$

\paragraph{Series} Dada la sucesion ${a_k}$, construimos la sucesion ${S_n}$ de la manera siguiente:
\begin{equation*}
	S_n = \sum_{k=0}^{n} a_k
\end{equation*}

A esta sucesion de sumas parciales la denominamos \textbf{serie} de terminos $a_k$ y se lo denota como $\sum a_k$. Diremos que la serie $\sum a_k$ es convergente si lo es la sucesion de las sumas parciales ${S_n}$.\\
\linebreak

Si la serie $\sum a_k$ es convergente, existe:

\begin{equation*}
	\lim_{n\rightarrow \infty} \sum_{k=0}^{n} a_k \Rightarrow \sum_{k=0}^{\infty} a_k
\end{equation*}

\paragraph{Definicion} Dada la sucesion ${a_k}$ en $\mathbb{C}$ y $z_0 \in \mathbb{C}$, la serie $\sum a_k (z-z_0)^k$ se denomina serie de potencias de coeficientes $a_k$ alrededor de $z_0$

\paragraph{Algebra de series}
Dadas las series $\sum a_k$ y $\sum b_k$ podemos construir las series que son sumas, restas, productos y cocientes de las mismas.

\begin{equation*}
\sum_{k \geq 0} a_k \pm \sum_{k \geq 0} b_k = \sum_{k \geq 0} (a_k \pm b_k)
\end{equation*}
 
\begin{equation*}
\left( \sum_{k \geq 0} a_k \right) \left( \sum_{k \geq 0} b_k \right) = \sum_{k \geq 0} \left( \sum_{j = 0}^k a_j b_{k j} \right) \text{Producto de cauchy}
\end{equation*}

\begin{equation*}
	\dfrac{\sum_{k\geq 0} a_k}{\sum_{k\geq 0} b_k} = \sum_{k\geq 0}c_k
\end{equation*}
Que despejando $\sum_{k\geq 0} a_k$ se puede obtener $c_k$ de manera recursiva.
\section{serie de Taylor y Laurent}

\paragraph{Serie de Taylor}

Sea $f: \mathbb{C} \rightarrow \mathbb{C}$ analitica, $z_0 \in dom(f)$ y sea $\Gamma$ una curva cerrada simple contenida en $dom(f)$, si $z_0$ pertenece a $\Gamma_0$ y $r > 0$ es tal que $Cr(z_0) \subset \Gamma_0$ entonces, para $z \in  Br(z_0)$

\begin{equation*}
	f(z) = \sum_{k\geq 0} \dfrac{f^{(k)} (z_0)}{k!} (z-z_0)^k
\end{equation*}

\paragraph{Serie de Log} Sea $f(z) = Log(z)$ y tomemos $c \in \mathbb{C}$ tal que $c \neq 0$ entonces

\begin{equation*}
	Log(z) = \sum_{k \geq 0} \dfrac{(-1)^k (z-1)^{k+1}}{k+1}
\end{equation*}

\paragraph{Serie de Laurent}

Sea $z_0 \in \mathbb{C}$ y seab $r$ y $R > 0$, si $f: \mathbb{C} \rightarrow \mathbb{C}$ es analitica en $A(z_0,r,R) = {z\in \mathbb{C} / r<|z-z_0|<R}$ entonces para $z\in A(z_0,r,R)$ tenemos

\begin{equation*}
	f(z) = \sum_{k \geq 0} a_k(z-z_0^k) + \sum_{k \geq 1} \dfrac{b_k}{(z-z_0)^k}
\end{equation*} 
Donde:

\begin{equation*}
	a_k = \frac{1}{2\pi i} \int_{CR(z_0)} \dfrac{f(w)}{(w-z_0)^{k+1}}dw = \dfrac{f^{(k)}(z_0)}{k!}
\end{equation*}

\begin{equation*}
	b_k = \frac{1}{2\pi i} \int_{Cr(z_0)} \dfrac{f(w)}{(w-z_0)^{-k+1}}dw
\end{equation*}

\paragraph{Residuo}
\begin{equation*}
	b_1 = \frac{1}{2\pi i} \int_{Cr(z_0)} f(w)dw \Rightarrow \int_{Cr(z_0)}f(w)dw = 2\pi i b_1
\end{equation*}
y se denota $b_1 = Res(f(z), z_0)$

\section{Clasificacion de puntos singulares}

Dada $f: \mathbb{C} \rightarrow \mathbb{C}$ y $z_0$ un punto singular de la misma, tenemos que para $z=z_0$

\begin{equation*}
	f(z) = \sum_{k\geq 0} a_k(z-z_0)^k + \sum_{k \geq 1} \dfrac{b_k}{(z-z_0)^k} 
\end{equation*}

Analizaremos los distintos casos



\end{document}

[ver funciones elementales y armonicas]