\documentclass[11pt]{article}

\usepackage{sectsty}
\usepackage{amsmath}
\usepackage{graphicx}

% Margins
\topmargin=-0.45in
\evensidemargin=0in
\oddsidemargin=0in
\textwidth=6.5in
\textheight=9.0in
\headsep=0.25in

\title{ Resumen Algebra 2$^{do}$ parcial}
\author{ Mateo P. Cetti }
\date{\today}

\begin{document}
\maketitle	
\pagebreak

% Optional TOC
% \tableofcontents
% \pagebreak

%--Paper--

\section{Sistemas de coordenadas}

\paragraph{Sistema de coordenadas}
Es una correspondencia entre los puntos de una recta y los numero reales. El numero "X" asociado a un punto en particular se llama \textbf{coordenada} en ese punto.

\paragraph{}
para formar un sistema de coordenadas, se necesita un par de rectas equipadas con sus escalas y construidas de manera tal que se corten (coordenada 0)

\paragraph*{}
Un sistema de coordenadas establece una \textbf{correspondencia biunívoca} entre los puntos del plano y los pares ordenados de números reales. Esta correspondencia permite identificar los puntos con los pares ordenados correspondientes
Los puntos sobre el primer eje tienen segunda coordenada nula, puntos sobre el segundo eje tienen primera coordenada nula.

\paragraph{Recta numerica}
Recta \emph{equipada} con un sistema de coordenadas.

\paragraph{Sistema de ejes cartesianos}
Sistema de coordenadas con ejes \textbf{Perpendiculares} y una misma unidad de medida en c/eje

\paragraph{Sistema de ejes en el espacio}
Se nececita de 3 ejes o rectas \textbf{no coplanares} que se intersecten en un punto. Este establece una \textbf{correspondencia biunivoca} entre los puntos del espacio y las ternas ordenadas de numeros reales

\paragraph{}
Las 2 primeras coordenadas (x,y) nos determinan la "sombra del punto".

\section{Vectores}

\paragraph{Segmento de recta}
Esta determinado por los puntos P (inicial) y Q (final), y esta representado por una flecha que une ambos puntos, con inicio en P y final en Q.
\paragraph{Vector}
Sean sus puntos P y Q, un vector se denota como $\overrightarrow{PQ}$ o con una letra minuscula (EJ: \emph{v}).
\paragraph{Elementos}
\begin{enumerate}
	\item \textbf{Modulo: } Longitud del segmento.
	\item \textbf{Triseccion: } (Inclinacion)Angulo que forma la recta que contiene el vector con el semieje \textbf{positivo} de las X.
	\item \textbf{Sentido: } Dado por la flecha (De P a Q, o de Q a P).
\end{enumerate}

\paragraph{Vectores equipolentes}
Vectores que tienen la misma longitud, direccion y sentido (Vectores cuyos elementos son iguales.)

\pagebreak

\paragraph{Vectores libres}
 Vector con origen en un punto X, determinado por sus \textbf{componentes} o numeros de direccion.
\begin{itemize}
	\item En R$^2$: (V$_x$, V$_y$)
	\item En R$^3$: (V$_x$, V$_y$, V$_z$)
\end{itemize}
\paragraph{}
Todos los vectores libres con mismos componentes son \textbf{equipolentes}, sin importar el punto origen X.

\paragraph{Correspondencia entre puntos y vectores}
Un par ordenado (a, b) representa:
\begin{enumerate}
	\item Un punto "P" de coordenadas (a,b).
	\item Un vector libre "v" de componentes (a,b) y origen en (0, 0).
\end{enumerate}

\paragraph{Operaciones con vectores libres}
\begin{enumerate}
	\item \textbf{Suma entre vectores:} Se suman los componentes \textbf{homologos} de los vectores sumandos.
	\textbf{Propiedades:}
	\begin{itemize}
		\item \textbf{Asociativa: } (a + b) + c = a + (b + c)
		\item \textbf{Conmutativa: } a + b = b + a
		\item \textbf{Elemento neutro: } a + (0,0) = a
		\item \textbf{Opuesto: } a+(-a) = 0 
	\end{itemize}
	\item \textbf{Producto de un vector por un escalar:} Se multiplica cada componente del vector por un escalar. k.a = (k.a$_x		$, k.a$_y$)\\
	\textbf{Propiedades:}
	\begin{itemize}
		\item \textbf{Distributiva (respecto a la suma de vectores): } k.(a + b) = k.a + k.b 
		\item \textbf{Distributiva (Respecto a la suma de escalares): } (k + k').a = k.a + k'.a
		\item \textbf{Asociativa: } (k.k').a = k.(k'.a) = k'.(k.a)
		\item \textbf{Elemento neutro: } 1.a = a
	\end{itemize}
\end{enumerate}

\paragraph{Vectores libres paralelos}
Dos vectores son paralelos si y solo si uno es multiplo escalar de otro. (a // b si y solo si b = k.a). \textbf{Nota:} El Vector nulo es paralelo a \textbf{TODO} vector.

\pagebreak

\section{Recta en R$^2$ y sus ecuaciones cartesianas}

\paragraph{Ecuación vectorial: }
Representa la resolucion de un sistema de ecuaciones compatibles \textbf{indeterminado}.\\ 
Esta determinada por dos puntos distintos.\\
Ej: (x, y) = (x$_1$, y$_1$) + t.(v$_x$, v$_y$)
\begin{itemize}
	\item (x, y) = Punto generico
	\item (x$_1$, y$_1$) = Punto que pertenece a la recta
	\item t = parametro
	\item (v$_x$, v$_y$) = Vector direccional
\end{itemize}

\paragraph{Ecuaciones parametricas}
\begin{equation*}
  \left\{
    \begin{array}{@{} l c @{}}
      x = x_1 +t.v_x \\
      y = y_1 +t.v_y
    \end{array}\right.
  \label{eq4}
\end{equation*}

\paragraph{Ecuacion simetrica} Esta ecuacion se puede obtener siempre y cuando v$_x$ y v$_y$ no sean 0. Se igualan los parametros t en cada ecuacion de las ecuaciones parametricas y se igualan

\begin{equation*}
	\frac{x-x_1}{v_x} = \frac{y-y_1}{v_y}
\end{equation*}

\paragraph{Ecuacion implicita}
\begin{equation*}
	\frac{x-x_1}{v_x} = \frac{y-y_1}{v_y}
\end{equation*}

\begin{equation*}
	(x-x_1)*v_y = (y-y_1) * v_x
\end{equation*}

\begin{equation*}
	x*v_y-x_1*v_y = y*v_x-y_1*v_x
\end{equation*}

\begin{equation*}
	x*v_y - y*v_x + (-x_1*v_y +y_1*v_x)= 0
\end{equation*}

\begin{equation*}
	Ax + By + C = 0
\end{equation*}

\paragraph{Ecuacion explicita}

\begin{equation*}
	Ax + By + C = 0 => -By = Ax+C
\end{equation*}

\begin{equation*}
	y = -\frac{A}{B} x- \frac{C}{B}
\end{equation*}

\begin{equation*}
	y = mx + b
\end{equation*}

\begin{itemize}
	\item \textbf{m:} inclinacion de la recta
	\item \textbf{b:} punto en que la recta intercepta al eje de coordenadas
\end{itemize}

\paragraph{Ecuacion segmentaria}
Se obtiene desde la ecuacion implicita

\begin{equation}
	Ax + By + C = 0 \rightarrow Ax + By + C - C = - C
\end{equation}

\begin{equation}
	- \dfrac{A}{C} x - \dfrac{B}{C} y  = 1
\end{equation}

\begin{equation}
	- \dfrac{\frac{x}{1}}{\frac{C}{A}} - \dfrac{\frac{y}{1}}{\frac{C}{B}}  = 1
\end{equation}

\begin{equation}
	\dfrac{x}{a} + \dfrac{x}{b}  = 1
\end{equation}

donde:

\begin{itemize}
	\item \textbf{a} = corte de la funcion con el eje x
	\item \textbf{b} = corte de la funcion con el eje y
\end{itemize}

La unica forma \textbf{CARTESIANA} de representar \textbf{CUALQUIER} recta en el plano es mediante la ecuacion implicita, ya que:

\begin{itemize}
	\item Si la recta es paralela al eje \textbf{X}, no se puede representar mediante la ecuacion \textbf{simetrica} ni mediante la ecuacion \textbf{segmentaria}.
	\item Si la recta es paralela al eje \textbf{Y}, no se puede representar mediante la ecuacion \textbf{simetrica}, \textbf{explicita}, ni tampoco la \textbf{segmentaria}.
\end{itemize}

\paragraph{Paralelismo e interseccion de rectas en R$^2$}

\begin{itemize}
	\item \textbf{Rectas paralelas: } Siempre que sus vectores sean paralelos ($r1 // r2 \Leftrightarrow v // u \Leftrightarrow v = k.u $)
	\begin{itemize}
		\item \textbf{No coincidentes: }no tienen ningun punto en comun (sistema incompatible [Ecuaciones parametricas])
		\item \textbf{Coincidentes: }comparten todos sus puntos (Sistema compatible indeterminado)
	\end{itemize}
	\item \textbf{Rectas concurrentes} Tienen solo un punto en comun (Sistema compatible determinado).
\end{itemize}

\section{Rectas en el espacio R$^3$}
\begin{itemize}
	\item Representa un conjunto solucion de un sistema de ecuaciones lineales compatible indeterminado.\\
	\item En R$^3$ no se puede definir la recta de forma \textbf{explicita} ni \textbf{segmentaria}\\
	\item Una recta en R$^3$ es la interseccion entre 2 planos.\\
\end{itemize}

\paragraph{Ecuacion vectorial}
\begin{equation*}
	(x, y, z) = (x_1, y_1, z_1) + t(v_x, v_y, v_z)
\end{equation*}

\paragraph{Ecuaciones parametricas}

\begin{equation*}
  \left\{
    \begin{array}{@{} l c @{}}
      x = x_1 +t.v_x \\
      y = y_1 +t.v_y \\
      z = z_1 +t.v_z
    \end{array}\right.
  \label{eq4}
\end{equation*}

\paragraph{Ecuaciones simetricas}

\begin{equation*}
	\dfrac{x-x_1}{v_x} = \dfrac{y-y_1}{v_y} = \dfrac{z-z_1}{v_z}
\end{equation*}

\paragraph{Ecuacion general o implicita}

\begin{equation*}
  \left\{
    \begin{array}{@{} l c @{}}
      A_1x + B_1y + C_1z = 0 \\
	A_2x + B-2y + C_2z = 0	
    \end{array}\right.
\end{equation*}

\paragraph{Paralelismo e interseccion entre rectas en R$^3$}
Se presentan los siguientes casos:
\begin{itemize}
	\item \textbf{Rectas paralelas}
	\begin{itemize}
		\item \textbf{Rectas coincidentes: } Sistema compatible indeterminado
		\item \textbf{Rectas no coincidentes: } Sistema incompatible
	\end{itemize}
	\item \textbf{Rectas NO paralelas}
	\begin{itemize}
		\item \textbf{Rectas concurrentes: }Sistema compatible determinado
		\item \textbf{Rectas alabeadas: }Sistema incopatible
	\end{itemize}
\end{itemize}
\paragraph{Rectas paralelas a los ejes coordenados}
Una recta es paralelas a un eje coordenado si su vector direccion tiene 2 componentes nulas.
\begin{itemize}
	\item \textbf{paralela al eje x:} v = (v$_x$, 0, 0)
	\item \textbf{paralela al eje y:} v = (0, v$_y$, 0)
	\item \textbf{paralela al eje z:} v = (0, 0, v$_z$)
\end{itemize}
\section{Plano}
Esta determinado por 3 puntos \textbf{NO COLINEALES}
\paragraph{Ecuacion vectorial del plano}
\begin{equation*}
	(x, y, z) = (x_1, x_1, x_1) + \alpha(u_x, u_y, u_z) + \beta(v_x, v_y, v_z)
\end{equation*}
\begin{itemize}
	\item $\alpha > 0$ y $\beta > 0 \rightarrow$ \textbf{Primer cuadrante}
	\item $\alpha < 0$ y $\beta > 0 \rightarrow$ \textbf{Segundo cuadrante}
	\item $\alpha > 0$ y $\beta < 0 \rightarrow$ \textbf{Tercer cuadrante}
	\item $\alpha < 0$ y $\beta < 0 \rightarrow$ \textbf{Cuarto cuadrante}
\end{itemize}
\paragraph{Ecuaciones parametricas}
\begin{equation*}
  \left\{
    \begin{array}{@{} l c @{}}
    x = x_1 + \alpha u_x + \beta v_x \\
    y = y_1 + \alpha u_y + \beta v_y\\
    z = z_1 + \alpha u_z + \beta v_z
    \end{array}\right.
\end{equation*}
\paragraph{Ecuacion cartesiana}
Se obtiene determinando la condicion que deben cumplir  X, Y y Z para que el sistema
de ecuaciones parametricas , cuyas incognitas son $\alpha$ y $\beta$ sea COMPATIBLE
\begin{equation*}
  \left\{
    \begin{array}{@{} l c @{}}
    x = x_1 + \alpha u_x + \beta v_x \\
    y = y_1 + \alpha u_y + \beta v_y\\
    z = z_1 + \alpha u_z + \beta v_z
    \end{array}\right.
\end{equation*}
\begin{equation*}
  \left\{
    \begin{array}{@{} l c @{}}
    x - x_1 = \alpha u_x + \beta v_x \\
    y - y_1 = \alpha u_y + \beta v_y\\
    z - z_1 = \alpha u_z + \beta v_z
    \end{array}\right.
\end{equation*}
\begin{equation*}
\begin{bmatrix}
	{u_x} & {v_x} & {\vert} &  {x - x_1}\\
	{u_y} & {v_y} & {\vert} &  {y - y_1}\\
	{u_z} & {v_z} & {\vert} &  {z - z_1}
\end{bmatrix}
\end{equation*}
\begin{equation*}
\begin{bmatrix}
	{1} & {0} & {\vert} &  {*}\\
	{0} & {1} & {\vert} &  {*}\\
	{0} & {0} & {\vert} &  {Ax + By + Cz + D}
\end{bmatrix}
\end{equation*}\\
\begin{equation*}
	Ax + By + Cz + D = 0
\end{equation*}
\paragraph{Paralelismo e interseccion entre planos}
2 planos pueden ser:
\begin{itemize}
	\item \textbf{Paralelos:}
	\begin{itemize}
		\item \textbf{Coincidentes} Tienen \textbf{TODOS} sus puntos en comun
		\item \textbf{No coincidentes} No tienen \textbf{NINGUN} punto en comun	
	\end{itemize}
	\item \textbf{No paralelos: } (Se intersectan) Tienen en comun los puntos de la recta r
		(interseccion entre ambos.)
\end{itemize}
\paragraph{Condicion de paralelismo de planos}
siendo:
\begin{itemize}
	\item plano1 ($\pi_1$) = P + $\alpha$u + $\beta$v
	\item plano2 ($\pi_2$) = P' + $\alpha$u' + $\beta$v'
\end{itemize}
Los vectores u y v son combinacion lineal de u' y v'.
\begin{equation*}
	u' = k_1.u + k_2.v
\end{equation*}
\begin{equation*}
	v' = k_3.u + k_4.v
\end{equation*}
\paragraph{Si usamos la ecuacion CARTESIANA}
$\pi_1 = A_1x + B_1y + C_1z + D_1 $ \\
$ \pi_2 = A_2x + B_2y + C_2z + D_2$ \\
$(A_1, B_1, C_1) = k(A_2, B_2, C_2)$
\pagebreak
\section{Plano y Recta en R$^3$}
\paragraph{Pociciones relativas de una recta con respecto a un plano}
\begin{itemize}
	\item \textbf{Paralelos}
	\begin{itemize}
		\item \textbf{La recta esta incluida en el plano} Sistema compatible indeterminado
		\item \textbf{La recta esta en incluida en un plano paralelo al plano} Sistema incompatible
	\end{itemize}
	\item \textbf{No paralelos} La recta corta al plano (Tienen un punto en comun)sistema compatible determinado.
\end{itemize}
\paragraph{Condicion de paralelismo entre recta y plano}
Siendo:
\begin{itemize}
	\item \textbf{recta r: } $(x, y , z) = (x_1, y_1, z_1) + t.(v_x, v_y, v_z)$
	\item \textbf{Plano $\pi$: }$(x, y, z) = (x_1, y_1, z_1) + \alpha(u_x, u_y, u_z) + \beta(w_x, w_y, w_z)$ 
\end{itemize}

r es paralelo a $\pi$ siempre que v sea una \textbf{combinacion lineal} de u y w.

\begin{equation*}
	(v_x, v_y, v_z) = k_1 (u_x, u_y, u_z) + k_2 (w_x, w_y, w_z)
\end{equation*}
\section{Problemas metricos en R$^2$ y R$^3$}

\paragraph{Longitud (Modulo o norma) de un vector libre}
Longitud de uno de los segmentos que representa al vector libre.\\
Siempre es positivo y se denota como $|| u ||$\\
Es la distancia entre el origen del sistema de coordenadas al extremo del segmento dirigido\\
Siendo $u = (u_x, u_y)$
\begin{equation*}
	||u|| = \sqrt[2]{(u_x)^2+(u_y)^2}
\end{equation*}
En R$^3$:
$v = (v_x, v_y, v_z)$
\begin{equation*}
	||u|| = \sqrt[2]{(v_x)^2+(v_y)^2 + (v_z)^2}
\end{equation*}
\paragraph{Angulo entre vectores}
Siendo u y v 2 vectores NO nulos en R$^2$ o R$^3$, el angulo definido entre ellos ($\theta$) siempre va a tener valores entre 0 y $\pi$
\begin{center}
	$0 \leq \theta \leq \pi$
\end{center}
\paragraph{Producto punto, producto interior  o producto escalar entre vectores.}
El resultado es un numero real que se obtiene de la siguiente formula:
\begin{equation*}
	u \cdot v = ||u||.||v||.cos \theta
\end{equation*}
Aunque usando el \textbf{Teorema del coseno} se obtiene:
\begin{itemize}
	\item en R$^2$
	\begin{equation*}
		u \cdot v = u_x . v_x + u_y . v_y
	\end{equation*}
	\item en R$^3$
	\begin{equation*}
		u \cdot v = u_x . v_x + u_y . v_y + u_z . v_z
	\end{equation*}
\end{itemize}
\paragraph{Propiedades del producto punto entre vectores}
\begin{enumerate}
	\item \textbf{Conmutativa} $u \cdot v = v \cdot u$
	\item \textbf{Distributiva respecto a la suma de vectores} $u \cdot (v + w) = u \cdot b + u \cdot w$
	\item \textbf{Asociativa} $k (u \cdot v) = (k.u) \cdot v = u \cdot (k.v)$
	\item $ u \cdot u = ||u||^2 $
	\item $||u|| = \sqrt[2]{u \cdot u}$
	\item \textbf{Positividad:} El modulo de un vector es un numero real positivo o nulo
\end{enumerate}
\paragraph{Distancia entre 2 puntos} 
Dos puntos P y Q definen un segmento de recta y un vector libre $u = Q - P$. la distancia
entre estos puntos (d(P,Q)) es igual a la longitud del vector u.
\begin{equation*}
	||u|| = d(P,Q)
\end{equation*}
(Esto se utiliza por ejemplo, para hallar la distancia entre un punto y un plano)
\paragraph{Propiedad sin nombre xD}
sea u el vector $u = (u_x, u_y, u_z)$ y c un escalar cualquiera,
\begin{equation*}
	||c.u|| = |c| . ||u||
\end{equation*}
\paragraph{Vectores unitarios}
Un vector es unitario cuando su modulo vale 1. Para transformar un vector cualquiera a unitario hacemos:
\begin{equation*}
	\dfrac{v}{||v||}
\end{equation*}
\paragraph{Angulo entre vectores}
\begin{equation*}
	u \cdot v = ||u||.||v||.cos \theta
\end{equation*}
\begin{equation*}
	cos \theta = \dfrac{u \cdot v}{||u||.||v||}
\end{equation*}
\begin{equation*}
	\theta = \arccos (\dfrac{u \cdot v}{||u||.||v||})
\end{equation*}
\begin{enumerate}
	\item Si $u \cdot v > 0$, entonces $0 \leq \cos \theta < \frac{\pi}{2}$ y por lo tanto $\theta$ va a ser un \textbf{Angulo agudo}
	\item si $u \cdot v < 0$, $\frac{\pi}{2} \leq \cos \theta < \pi$ y por lo tanto $\theta$ va a ser un \textbf{Angulo obtuso}
	\item Si $u \cdot v = 0$, entonces $\cos \theta = 0$ y por lo tanto $\theta$ va a ser un angulo \textbf{Recto}.
	En este caso decimos que los vectores son \textbf{Ortogonales} ($\perp$) o perpendiculares entre si.
\end{enumerate}
\paragraph{Condicion de perpendicularidad entre vectores}
Siendo u y v vectores no nulos, estos seran perpendiculares (ortogonales entre si) siempre y cuando.
\begin{equation*}
	u \cdot v = 0
\end{equation*}
%--/Paper--

\end{document}