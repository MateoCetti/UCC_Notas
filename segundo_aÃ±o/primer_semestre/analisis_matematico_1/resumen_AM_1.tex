\documentclass[11pt]{article}

\usepackage{sectsty}
\usepackage{amsmath}
\usepackage{graphicx}

% Margins
\topmargin=-0.45in
\evensidemargin=0in
\oddsidemargin=0in
\textwidth=6.5in
\textheight=9.0in
\headsep=0.25in

\title{ Resumen Analisis matematico 2$^{do}$ parcial}
\author{ Mateo P. Cetti }
\date{\today}

\begin{document}
\maketitle	
\pagebreak

% Optional TOC
% \tableofcontents
% \pagebreak

%--Paper--

\section{Derivadas}
Funcion \textbf{continua: }No presenta saltos / cortes\\
Funcion \textbf{derivable: } No presenta picos / cambios bruscos
\textbf{Derivada}
\begin{itemize}
	\item Tasa de cambio \textbf{instantanea} de una magnitud
	\item \textbf{Pendiente} de la \textbf{recta tangente} a la grafica en un \textbf{punto}
\end{itemize}
\section{Funciones crecientes y decrecientes}
Siempre que $f(x)$ sea \textbf{continua} en un intervalo [a, b], esta sera:
\begin{enumerate}
	\item \textbf{Creciente: }si a \textbf{mayores} valores de la variable hay \textbf{mayores} valores de la funcion
	\item \textbf{Decreciente: }si a \textbf{mayores} valores de la variable hay \textbf{menores} valores de la funcion
\end{enumerate}
La funcion \textbf{EN UN PUNTO} tambien puede ser:
\begin{enumerate}
	\item \textbf{Creciente: }si la derivada de la funcion valuada en ese punto es  \textbf{mayor} a 0
	\item \textbf{Decreciente: }si la derivada de la funcion valuada en ese punto es \textbf{menor} a 0
	\item \textbf{Puntos estacionarios: }si la derivada de la funcion valuada en ese punto es 0
\end{enumerate}
\section{Rolle}
Sea $f(x)$ continua en un intervalo \textbf{cerrado} [a, b] y derivable en \textbf{todos} sus puntos, y tambien $sea f(a) = f(b)$ \textbf{entonces} existe un punto c tal que $x=c$ y $f'(c) = 0$
\section{Lagrange}
sea $f(x)$ continua y derivable en [a, b], existe un punto c tal que $a < c < b$ donde:
\begin{equation}
	f(b)-f(a) = f'(c).(b-a)
\end{equation}
\begin{equation}
	\dfrac{f(b)-f(a)}{b-a} = f'(c)
\end{equation}
\section{Extremos}
\paragraph{Absolutos: } sea $f(x)$ continua en [a ,b] habra un punto $x_1$ y $x_2$ tal que $f(x_1) > f(x)$ y $f(x_2) < f(x)$ siendo x un punto \textbf{CUALQUIERA} de la funcion en el intervalo.\\
Esto no siempre funciona en intervalos \textbf{abiertos}
\paragraph{Teorema: }Sea $f(x)$ continua en [a,b] y a y b tienen distinto signo, hay un punto $a < c < b$ en donde $f(c) = 0$
\paragraph{Extremos relativos} sea $f(x)$ continua en un entorno abierto (a, b), habra:
\begin{itemize}
	\item \textbf{Maximo relativo:} si $f(x_1-\bigtriangleup x)<f(x_1)> f(x_1+\bigtriangleup x)$
	\item \textbf{Minimo relativo:} si $f(x_2-\bigtriangleup x)>f(x_2)< f(x_2+\bigtriangleup x)$
\end{itemize}
Siendo $\bigtriangleup x$ un valor arbitrariamente pequeño.
\paragraph{Condicion necesaria}para la existencia de extremos relativos (pero no \textbf{SUFICIENTE})\\
Habra un extremo en $x_1$ si $f'(x_1) = 0$. Este punto se denomina \textbf{PUNTO CRITICO}
\paragraph{Condicion suficiente} para la existencia de extremos relativos\\
Se puede obtener mediante 2 metodos (En ambos se tiene que cumplor la condicion \textbf{necesaria}):
\begin{itemize}
	\item \textbf{Derivada primera} (Ya lo sabes $>:V$)
	\item \textbf{Derivada segunda} (Este tambien $>:V$)
\end{itemize}
\section{Concavidad}
$f''(x)$ es la razon de cambio de la \textbf{pendiente} de $f(x)$\\
si $f(x)$ es continua y derivable hasta el segundo orden en [a, b]:
\begin{itemize}
	\item Habra concavidad \textbf{positiva} si $f''(x) > 0$ en todo [a, b]
	\item Habra concavidad \textbf{Negativa} si $f''(x) < 0$ en todo [a, b]
\end{itemize}
\paragraph{Puntos de INFLEXION} Punto en una curva donde cambia el el signo de concavidad.\\
si $f(x)$ es continua y derivable hasta el orden 2 en [a, b]
\begin{itemize}
	\item Habra un punto de inflexion si $f''(x) = 0$
	\item si $f'(x) = 0$ entonces es un  punto de inflexion a \textbf{TANGENTE HORIZONTAL}
	\item Si $f''(x)$ no existe, pero a valores muy pequeños por izq y por derecha $f''(x)$ cambia de signo, entonces el punto de inflexion \textbf{SI EXISTE}
\end{itemize}
\end{document}