\title{Resumen de programacion}

\date{\today}

\documentclass[10pt]{article}

\usepackage{amsmath}
\usepackage{amsfonts}

\begin{document} 
\maketitle

\section{Recursion}

\paragraph{definición inductiva: } obtener conceptos nuevos empleando el 
mismo concepto que intenta describir.\\

Una \textbf{definición recursiva} dice cómo obtener conceptos
nuevos empleando el mismo concepto que intenta definir\\

Un razonamiento recursivo tiene \textbf{dos partes:} 
\begin{itemize}
	\item  La base, que no es recursiva y es el punto tanto de partida como de terminación de la definición. ( son los elementos del conjunto se especifican explícitamente y aseguren el final o corte)
	\item regla recursiva de construcción. Los elementos del conjunto que se definen en términos
de los elementos ya definidos
\end{itemize}

\paragraph{Recursividad de cola: } Cuando una llamada recursiva es la última posición ejecutada
del procedimiento

\paragraph{Recursion directa: }Cuando un procedimiento incluye una llamada a si mismo

\paragraph{Recursion indirecta: }Cuando un procedimiento llama a otro procedimiento y este
causa que el procedimiento original sea invocado

Es menos peligrosa una recursión infinita que un ciclo infinito,
ya que una recursividad infinita pronto \textbf{se queda sin espacio} (para cada llamada se 
crean copias independientes de las variables declaradas en el procedimiento) y termina el
programa, mientras que la iteración infinita puede continuar mientras no se termine en forma manual.\\
\textbf{el cálculo recursivo es mucho más caro que el iterativo.}\\
\linebreak
\paragraph{Pasos de la recursion:}
\begin{enumerate}
	\item El procedimiento se llama a si mismo
	\item El problema se resuelve, resolviendo el mismo problema pero de tamaño menor
	\item La manera en la cual el tamaño del problema disminuye asegura que el caso base eventualmente se alcanzará
\end{enumerate}

\end{document}